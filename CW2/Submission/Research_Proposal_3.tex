% !TEX TS-program = pdflatex
% !TEX encoding = UTF-8 Unicode

% This is a simple template for a LaTeX document using the "article" class.
% See "book", "report", "letter" for other types of document.

\documentclass[11pt]{article} % use larger type; default would be 10pt

\usepackage[utf8]{inputenc} % set input encoding (not needed with XeLaTeX)
\usepackage[backend=bibtex8, style=authoryear]{biblatex}
\addbibresource{bibliography.bib}
%%% Examples of Article customizations
% These packages are optional, depending whether you want the features they provide.
% See the LaTeX Companion or other references for full information.

%%% PAGE DIMENSIONS
\usepackage{geometry} % to change the page dimensions
\geometry{a4paper} % or letterpaper (US) or a5paper or....
% \geometry{margin=2in} % for example, change the margins to 2 inches all round
% \geometry{landscape} % set up the page for landscape
%   read geometry.pdf for detailed page layout information

%\usepackage{graphicx} % support the \includegraphics command and options

\usepackage[parfill]{parskip} % Activate to begin paragraphs with an empty line rather than an indent

%%% PACKAGES
\usepackage{booktabs} % for much better looking tables
\usepackage{array} % for better arrays (eg matrices) in maths
\usepackage{paralist} % very flexible & customisable lists (eg. enumerate/itemize, etc.)
\usepackage{verbatim} % adds environment for commenting out blocks of text & for better verbatim
\usepackage{subfig} % make it possible to include more than one captioned figure/table in a single float
\usepackage{graphicx}
% These packages are all incorporated in the memoir class to one degree or another...

%%% HEADERS & FOOTERS
\usepackage{fancyhdr} % This should be set AFTER setting up the page geometry
\pagestyle{fancy} % options: empty , plain , fancy
\renewcommand{\headrulewidth}{0pt} % customise the layout...
\lhead{}\chead{}\rhead{}
\lfoot{}\cfoot{\thepage}\rfoot{}

%%% SECTION TITLE APPEARANCE
\usepackage{sectsty}
\allsectionsfont{\sffamily\mdseries\upshape} % (See the fntguide.pdf for font help)
% (This matches ConTeXt defaults)

%%% ToC (table of contents) APPEARANCE
\usepackage[nottoc,notlof,notlot]{tocbibind} % Put the bibliography in the ToC
\usepackage[titles,subfigure]{tocloft} % Alter the style of the Table of Contents
\renewcommand{\cftsecfont}{\rmfamily\mdseries\upshape}
\renewcommand{\cftsecpagefont}{\rmfamily\mdseries\upshape} % No bold!

%%% END Article customizations

%%% The "real" document content comes below...

\title{\vspace{-3.0cm}SugarScape}
%\date{} % Activate to display a given date or no date (if empty),
         % otherwise the current date is printed 

\begin{document}
\maketitle

\section{Initial Observations}

First, after an initial adjustment phase, where the population decreases, the number of turtles reaches an equillibrium. 
Second, the average values for the turtle variables behave consistently over model runs and initial population values. Average vision increases and average metabolism decreases until reaching steady states. Since this is nly affected by the death of turtles, equillibrium for all model variables is reached simultaneously. 

\section{Setting up the experiment}

First, the maximum number of ticks required to reach steady state was found. The model with populations from 10 to 1000 incrementing by 10, and running each popluation value 10 times, for 910 total runs, for 200 ticks. It was found in the results that the number of turtles change after the 26th tick for any trial. Additionally, it was observed that larger initial populations took longer on average to reach a constant population. 

%INSERT CHART OF POP BEHAVIOR HERE


With this knowledge, the experiment was set up to run 27 ticks for populations from 10 to 1000

The reporters used were average metabolism at each tick, average vision at each tick, and count at each tick. 

% INSERT CODE SNIPPET HERE

\subsection{Results}




XXXX words excluding headings, figures, and references. \\






\end{document}
